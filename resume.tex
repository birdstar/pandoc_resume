% Copyright 2013 Christophe-Marie Duquesne <chmd@chmd.fr>
% Copyright 2014 Mark Szepieniec <http://github.com/mszep>
% 
% ConText style for making a resume with pandoc. Inspired by moderncv.
% 
% This CSS document is delivered to you under the CC BY-SA 3.0 License.
% https://creativecommons.org/licenses/by-sa/3.0/deed.en_US

\startmode[*mkii]
  \enableregime[utf-8]  
  \setupcolors[state=start]
\stopmode
\mainlanguage[zh_CN]

\setupcolor[hex]
\definecolor[titlegrey][h=757575]
\definecolor[sectioncolor][h=397249]
\definecolor[rulecolor][h=9cb770]

% Enable hyperlinks
\setupinteraction[state=start, color=sectioncolor]

\setuppapersize [A4][A4]
\setuplayout    [width=middle, height=middle,
                 backspace=20mm, cutspace=0mm,
                 topspace=10mm, bottomspace=20mm,
                 header=0mm, footer=0mm]

%\setuppagenumbering[location={footer,center}]

\setupbodyfont[11pt, helvetica]

\setupwhitespace[medium]

\setupblackrules[width=31mm, color=rulecolor]

\setuphead[chapter]      [style=\tfd]
\setuphead[section]      [style=\tfd\bf, color=titlegrey, align=middle]
\setuphead[subsection]   [style=\tfb\bf, color=sectioncolor, align=right,
                          before={\leavevmode\blackrule\hspace}]
\setuphead[subsubsection][style=\bf]

\setuphead[chapter, section, subsection, subsubsection][number=no]

%\setupdescriptions[width=10mm]

\definedescription
  [description]
  [headstyle=bold, style=normal,
   location=hanging, width=18mm, distance=14mm, margin=0cm]

\setupitemize[autointro, packed]    % prevent orphan list intro
\setupitemize[indentnext=no]

\setupfloat[figure][default={here,nonumber}]
\setupfloat[table][default={here,nonumber}]

\setuptables[textwidth=max, HL=none]

\setupthinrules[width=15em] % width of horizontal rules

\setupdelimitedtext
  [blockquote]
  [before={\setupalign[middle]},
   indentnext=no,
  ]


\starttext

\section[王鹏]{王鹏}

\thinrule

\startitemize[packed]
\item
  入职IBM的9年时间,一直从事研发工作,涉猎广泛,具有比较开发的开发经验,具有快速的学习能力,为人勤奋谦逊,喜欢研究新事物。
\item
  2008.4 - 2015.4,从事DB2数据库工具的开发。
\item
  2015.4 -
  2017.6,从事IBM云计算平台Bluemix的监控系统,自动化运维系统和容器服务的开发。
\item
  2017.6-至今,从事大型主机机器学习平台产品的开发。
\stopitemize

\thinrule

\subsection[个人信息]{个人信息}

\startdescription{邮箱}
  {\bf birdstar@163.com}
\stopdescription

\startdescription{手机号}
  {\bf 15810292708}
\stopdescription

\startdescription{出生年月}
  {\bf 1982年9月}
\stopdescription

\startdescription{学历}
  {\bf 硕士研究生}
\stopdescription

\startdescription{籍贯}
  {\bf 山东青岛}
\stopdescription

\subsection[教育背景]{教育背景}

\startdescription{2005.9 - 2008.4}
  {\bf 工学硕士, 计算机软件与理论} 大连海事大学(保送)
\stopdescription

\startdescription{2001.9 - 2005.6}
  {\bf 工学学士, 计算机科学与技术} 大连海事大学
\stopdescription

\subsection[工作经历]{工作经历}

\startdescription{2008.4 - 至今}
  {\bf IBM 中国软件开发中心(CDL)}
\stopdescription

\subsection[项目经验]{项目经验}

{\bf 2017.6-至今: IBM Machine Learning for zOS}
为数据科学家提供的一套基于SparkML的机器学习开发平台,包括数据导入,模型训练,部署,预测。
https://www.ibm.com/us-en/marketplace/machine-learning-for-zos

\startitemize
\item
  个人角色和职责:技术负责人之一,主要看在线编辑器,运行时管理和集成,另外还有整个产品基于k8s的整体部署。
\item
  技术领域:Jupyter, Livy, KernelGateway, Toree, Spark,kubernetes。
\stopitemize

{\bf 2016.6-2017.6: 云平台容器服务的构建}
基于k8s为云平台上的服务提供容器化部署解决方案

\startitemize
\item
  个人角色和职责:技术负责人之一,从零开始搭建整个系统,重点负责存储分系统的设计实现部署,性能测试调优等。
\item
  技术领域:kubernetes,nfs,iscsi,glusterfs。
\stopitemize

{\bf 2016.2-2016.6: 智能自动化运维机器人构建} 基于slack
bot和监控报警系统,利用watson nlp,
自动处理原因明确的报警,对不明确的推荐解决方案。

\startitemize
\item
  个人角色和职责:技术负责人,全面负责设计实现部署上线。
\item
  技术领域:kafka,watson,slack api,ansible
\stopitemize

{\bf 2015.4-2016.2: 云平台监控报警系统构建 &
数据分析部云服务的运维(devops)}
提供对主机,服务,网络进行端到端的监控。

\startitemize
\item
  个人角色和职责:技术负责人之一,主要负责开发服务uptime和端到端监控,同时轮值运维我们部门(数据分析部)的云服务。。
\item
  技术领域:collectd, graphit, granfana, uptime, pagerduty, slack
\stopitemize

{\bf 2011.12-2015.4: DB2开发工具Data Studio产品的开发} Data
Studio是基于Eclipse的DB2的管理和开发工具,和DB2数据库一起发布。

\startitemize
\item
  个人角色和职责:核心开发人员,各个模块都有所涉及,后期成为分模块的技术负责人,主要负责例程调试器(Routine
  Debugger)的设计和开发。
\item
  技术领域:DB2, OSGi, Eclipse plugin
\stopitemize

{\bf 2008.5-2011.12: DB2性能监控工具Optim Performance
Manager(OPM)产品的开发}
OPM是一个基于Web的数据库性能监控软件,监控DB2的内存,CPU,死锁,低效SQL等等。

\startitemize
\item
  个人角色和职责:开发人员,前期主要做前台UI开发,后期前后台都做,减少沟通成本。
\item
  技术领域:DB2, Flex,Cairngorm,Spring
\stopitemize

{\bf Your Most Recent Work Experience:}

Short text containing the type of work done, results obtained, lessons
learned and other remarks. Can also include lists and links:

\startitemize
\item
  First item
\item
  Item with \useURL[url1][http://www.example.com][][link]\from[url1].
  Links will work both in the html and pdf versions.
\stopitemize

{\bf That Other Job You Had}

Also with a short description.

\subsection[technical-experience]{Technical Experience}

\startdescription{My Cool Side Project}
  For items which don't have a clear time ordering, a definition list
  can be used to have named items.

  \startitemize[packed]
  \item
    These items can also contain lists, but you need to mind the
    indentation levels in the markdown source.
  \item
    Second item.
  \stopitemize
\stopdescription

\startdescription{Open Source}
  List open source contributions here, perhaps placing emphasis on the
  project names, for example the {\bf Linux Kernel}, where you
  implemented multithreading over a long weekend, or {\bf node.js} (with
  \useURL[url2][http://nodejs.org][][link]\from[url2]) which was
  actually totally your idea\ldots{}
\stopdescription

\startdescription{Programming Languages}
  {\bf first-lang:} Here, we have an itemization, where we only want to
  add descriptions to the first few items, but still want to mention
  some others together at the end. A format that works well here is a
  description list where the first few items have their first word
  emphasized, and the last item contains the final few emphasized terms.
  Notice the reasonably nice page break in the pdf version, which
  wouldn't happen if we generated the pdf via html.

  {\bf second-lang:} Description of your experience with second-lang,
  perhaps again including a
  \useURL[url3][https://github.com/githubuser/superlongprojectname][][link]\from[url3],
  this time placing the url reference elsewhere in the document to
  reduce clutter (see source file).

  {\bf obscure-but-impressive-lang:} We both know this one's pushing it.

  Basic knowledge of {\bf C}, {\bf x86 assembly}, {\bf forth},
  {\bf Common Lisp}
\stopdescription

\subsection[extra-section-call-it-whatever-you-want]{Extra Section, Call
it Whatever You Want}

\startitemize
\item
  Human Languages:

  \startitemize[packed]
  \item
    English (native speaker)
  \item
    ???
  \item
    This is what a nested list looks like.
  \stopitemize
\item
  Random tidbit
\item
  Other sort of impressive-sounding thing you did
\stopitemize

\stoptext
